\begin{chapquote}{Someone said this}
``This was said.''
\end{chapquote}

Dr. Denise Rangel Tracy gave a guest lecture on Monday.

\section{Adding Two Points on an Elliptic Curve}

Let $E=x^3+Ax+B$, where $A, B\in F$.
\\Let $P_{1}, P_{2}$ be two points on $E$. 
\\Let $P_{3}=P_{1} \oplus P_{2}$.

Algorithm Addition
\\(1) If $P_{1}=0$, then $P_{1} \oplus P_{2} = P_{2}$.
\\(2) Otherwise, write $P_{1}=(x_{1}, y_{1}), P_{2}=(x_{2}, y_{2})$.
\\(3) If $x_{1}=x_{2}$, and $y_{1}=-y_{2}$, then $P_{1}+P_{2}=\mathcal{O}$.
\\(4) Define:

$$ \lambda =\begin{cases} \frac{y_{2}-y_{1}}{x_{2}-x_{1}} & \text{ if } P_{1}\neq P_{2} \\ \frac{3x_{1}^2+A}{2y_{1}} & \text{ if } P_{1}=P_{2} \end{cases}$$

$$x_{3} = \lambda ^2 - x_{1} - x_{2}$$
$$y_{3}=\lambda(x_{1}-x_{3})-y_{1}$$
then $P_{1} \oplus P_{2} = (x_{3},y_{3})$.

Extended Euclidean Algoritnm 
\\$a,b \in \mathbb{Z}^{*}$, then there exists, $u,v \in \mathbb{Z}$ such that $au+bv=gcd(a,b)$. \\If $gcd(a,b)=1$, then $au+bv=1$ or $au=1-bv$. 
\\In finite fields,$\\
F_p=$ finite field order $p$ or Group $\mathbb{Z}_p$
\\$au \equiv 1$ (mod b)$\\
\Rightarrow a= \frac{1}{u}$\\


Lenstra's Elliptic Curve factorization Method (ECM)
\\Main idea: Pretend a group is a field and randomly try to find reciprocals. If you can't, then you found factor of A and B. 
\begin{example} 
$y= x^{3}+3x+7$ in $\mathbb{Z}_{187}$
\\Goal: Factor 187 .
\\Given $P=(38,112)$, computer $2P$.
\\
$$\lambda_{2p} = \frac{3(38)^{2}+3}{2(112)}=\frac{4335}{224}\equiv4335(91)\equiv34(91)\equiv3094\equiv102\mod{187}$$
\\
Need $\frac{1}{224}=91\mod{187}$ (use the Extended Euclidean Algorithm). 
\\
$$x_{2p}=43$$
$$y_{2p}=126\mod{187}$$
$$2P = (43, 126)$$
\\
Similarly, find $3P$.
$$3P=2P \bigoplus P$$
$$\lambda_{3p}=\frac{y_{2p}-y_{p}}{x_{2p}-x_{p}}=\frac{14}{15}\mod{187}$$
\\Need to find $\frac{1}{15}$ mod 187.
$$\gcd(15,187)=1$$     
$$3P=(54,105)$$
$$75*\frac{1}{15}=1$$
$$3P=(?,?)$$
Find $5P$. 
$$5P=3P\bigoplus 2P.$$
$$\lambda_{5p}=\frac{105-126}{54-43}=\frac{-21}{11}$$
\\Find $\frac{1}{11}\equiv ?\mod{187}$.
$$\gcd(11,187)=1$$
$\Rightarrow 11$ doesn't have an multiplicative inverse.
\\
$5P= ?$ Great thing. Found a factor of 187! 
\\
\\If instead, we find $5P$ over $\mathbb{Z}_{11}$, then: 
$$P=(5,2)$$
$$5P \equiv 0 \mod{11}$$
\\At some point, we tried to divide by "0."\\
$$y^{2}=x^{3}+Ax+B$$
\end{example}

Algorithm:
\\(1) Choose random $A,a,b$ all mod $N$
\\"N is the number you want to factor."
\\(2) $P=(a,b), B=b^{2}-a^{3}-Aa$
\\then,
\\ $y^{2}=x^{3}+Ax+B$
\\(3) Computer $n!P$\\
\indent (i) work, found $n!P$\\
\indent (ii) fail since, with $gcd ({\_}, N)=N$.\\
\indent (iii) $gcd ({\_},N) \neq 1$.\\
\indent Failure to invert ${\_}$ implies $gcd({\_},N)$ divide $N$ and thus found a factor of $N$.

\section{Elliptic Curves}
Elliptic Curves over finite fields:
\\Notation: $\mathbb{Z} / n \mathbb{Z}$: the set of integer modulo $n$.
$$\mathbb{Z} / n \mathbb{Z}={0\mod{n}, 1\mod{n},..., n-1)\mod{n}}.$$
We've defined addition and multiplication on this set:
$$a\mod{n}+b\mod{n}\equiv(a+B)\mod{n}$$
$$(a\mod{n})(b\mod{n})\equiv(ab)\mod{n}$$

$(\mathbb{Z} / n \mathbb{Z}, +)$ is an abelian group:
\begin{itemize}
\item Closed under addition.
\item Additive identity, $0\mod{n}$
\item Additive inverses: $-(a\mod{n})\equiv(n-a)\mod{n}$.
\item Associative and commutative. 
\end{itemize}

$(\mathbb{Z} / n \mathbb{Z}, *)$ is almost a group:
\begin{itemize}
\item Closed under multiplication.
\item Multiplicative identity, $1\mod{n}$
\item Associative and commutative.
\item \underline{But} lacks inverses!\\
\indent $(n\mod{n})^-1$ doesn't exist.\\
\indent If $n$ is composite, e.g. $n=pk$ where $p$ is prime, then $(k\mod{n})$ and $(p\mod{n})$ don't have inverses.
\end{itemize}

Why is that true?\\
\indent Suppose they have inverses, and $(k\mod{n})^-1$ exists. 
$$0\mod{n}\equiv pk\mod{n} \equiv (p\mod{n})(k\mod{n})$$
\\Multiplying the above by $(k\mod{n})^{-1}$ yields:
$$0\mod{n}\equiv p\mod{n}$$
\\This is a contradiction because $p|n, n\nmid p$.\\

When $p$ is prime, $((\mathbb{Z} / p \mathbb{Z})$ / $\{0\mod{p}\}, *)$ \underline{is} an abelian group. Denoted as $(\mathbb{Z}$/$p\mathbb{Z})^*$\\

For number theorists, 
$$\mathbb{F}_{p}=(\mathbb{Z}/p\mathbb{Z}, +, *)$$
\begin{itemize}
\item $(\mathbb{F}_{p}, +)$ is an abelian group.
\item $(\mathbb{F}_{p}$ \ $\{0\}, *)$ is an abelian group.
\item Multiplication distributes over addition:\\
$$a(b+c)\mod{p}=(ab)\mod{p}+(ac)\mod{p}$$
\end{itemize}
\begin{example}

Define elliptic curve $E: y^{2}=x^{3}+2x+1$ over $\mathbb{F}_{5}= \{0\mod{5},1\mod{5},...,4\mod{5}\}$
\\
\\Point $({x}_{0}, {y}_{0})$ on $E$:
\\${y}_{0}^{3} \equiv {x}_{0}^{3}+2{x}_{0}+1\mod{5}$
\end{example}

Points on $E$:
\\
\\\begin{tabular}{r|c|l}
x & $y^{2}$ & $(x,y)$ pairs\\
\hline
0 & 1 & $(0,1), \, (0,4)$ \\
1 & 4 & $(1,2),  \, (1,3) $\\
2 & 3 & None \\
3 & 4 & $(3,2), \,  (3,3)$\\
4 & 3 & None \\
\end{tabular}
\\

$$1 \times 1 \mod{5} = 1\\$$
$$2 \times 2 \mod{5} = 4\\$$
$$3 \times 3 \mod{5} = 4\\$$
$$4 \times 4 \mod{5} = 1\\$$

By the above calculations, we know if $y^2 = 1$, then $y = 1, 4$ and if $y^2 = 4$, then $y = 2, 3$. Therefore, there are two $(x, y)$ pairs when $x = 0, 1, 3$. We cannot find any $(x, y)$ pairs when $x = 2$ and $x = 4$ because $y^2 = 3$ does not have a square root $y$ in mod 5.\\

\textbf{Addition of Points}
\\
\\\begin{tabular}{r|cll}
0 & (0,1) & (0,4) & (1,2) $\cdots$\\
\hline
(0,1) & & $\mathcal{O}$ &\\
(0,4) & $\mathcal{O}$ & & \\
(1,2) & & & \\
\end{tabular}
\\
\\

If $P = ({x}_{1}, {y}_{1}), Q = ({x}_{2}, {y}_{2})$
and $P 
\neq-Q$, then define
$$ \lambda =\begin{cases} (y_{2}-y_{1}){(x_{2}-x_{1})}^{-1} &  P\neq Q \\ (3x_{1}^2+A){(2y_{1})}^{-1} &  P = Q \end{cases}$$
and ${x}_{3} = {\lambda}^{2}-{x}_{1}-{x}_{2}$, ${y}_{3} = \lambda({x}_{1}-{x}_{3})-{y}_1$, $P+Q=({x}_{3}, {y}_{3})$

\begin{example} Find $P + Q$ where $P = (0,1)$ and $Q = (0,3)$ are points on the elliptic curve $E: \, \, y^{2}=x^{3}+2x+1$

First, we notice that $P = (0,1) \not = Q = (1,3)$ and $P \not = -Q$, so we calculate $\lambda$:
\[\lambda \equiv (y_{2}-y_{1}){({x}_{2}-x_{1})}^{-1} \equiv 2{1}^{-1} \equiv 2\]
and then find that ${x}_{3} \equiv 2^2 - 0 - 1 \equiv 3$, ${y}_{3} \equiv 2(-3) - 1 \equiv 3$.
Therefore $P + Q = (3,3)$.
\end{example}

\begin{example} On the same curve, we find $(0,1) + (0,1)$ by first calculating $\lambda$: 
\[\lambda \equiv (3*{0}^{2}){(2*1)}^{-1}\] so that 
${x}_{3} \equiv 1^2 - 0 - 0 \equiv 1$ and ${y}_3 \equiv 1(0 - 1) - 1 \equiv 3$, and hence $(0,1) + (0,1) = (1,3)$. \end{example}
%\\\underline{Proposition}: For any point P on E,
%\\5P=P+P+P+P+P=0.
%\\If P is not 0, then for all 1\leq n \leq 4,
%\\\indent nP \neq 0.$
%\\This can be generalized to all  $\mathbb{F}_{p}$ for primes p.
%\\
%\\This proposition leads to a primality test:
%\\Input: large odd number n
%\\Sketch of algorithm:
%\\\indent Define an elliptic curve E over  $\mathbb{Z} / n \mathbb{Z}$.
%\\\indent Start adding points:
%\\\indent \indent $P+P$ (Is this 0?)
%\\\indent \indent $P+P+P$ 
%\\
%\\Where elliptic curves are used:
\begin{itemize}
\item Primality test/factoring algorithm 
\item Elliptic Diffie-Hellman
\item Digital signature
\end{itemize}





