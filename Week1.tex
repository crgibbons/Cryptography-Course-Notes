\begin{chapquote}{Prof. Gibbons, Class}
``If you don't like fun, you should probably drop this class.''
\end{chapquote}

We begin with a discussion of the fundamentals of \gls{cryptography}.  The first example we saw was a \gls{scytale}, which is a type of \gls{transposition}.  We also noted that the encrypted syllabus was encrypted using a \gls{substitution} cipher (specifically, a 13-\gls{Caesar shift}).  In fact, the syllabus also included a little bit of \gls{steganography}; the link to the actual syllabus was hidden using white text on white background.

\section{Important vocabulary}
 
Please see the glossary for a list of the important vocabulary from the reading.

\section{Conventions}
To distinguish between enciphered and deciphered text, we use the following conventions.  Plain text is written in lower case: \plain{This is an example of plain text.}  
Enciphered text is written in upper case.  Enciphering the example sentence using a 13-Caesar shift, we see that our enciphered message reads:
\EdefRot{13}\ex{THIS IS AN EXAMPLE OF PLAIN TEXT}
{\tt \ex}.


\section{What Makes a Good Code}
After class discussion, we decided that a good encryption system should satisfy some basic criteria:
\begin{itemize}
\item Hard to crack
\item Reasonably easy to decrypt
    \begin{itemize}
    \item Key and algorithm for en- or decryption available to the person en- or decrypting (securely transmitted)
    \item Some system where the person decrypting doesn't need a key securely transmitted
    \end{itemize}
\item Key is easy to look up if you know what to do, but hard to guess (lots of possibilities)
\item Potentially use layers of encryption for extra security
\item Takes a reasonable amount of time and/or physical resources (like a code book) to en- or decrypt a message (relative to the scenario in which the message is being en- or decrypted)
\end{itemize}

We note that a system that requires sending someone a key securely puts us in an infinite loop of encrypting a message, then encrypting its key, then encrypting the key's key, \textit{ad nauseum}.  It would be nice to find a way around this.  One option would be for both parties to meet up and exchange relevant info (or hire/build a third party to do so).  However, this may put us at risk of violating the last principle of our ``good encryption'' list!

For example, if Amazon.com has to securely transmit a key to each customer (who must then encrypt and send their payment info to Amazon.com), they have to expend a huge amount of time and money to set up the meetings between Amazon.com representatives and customers.  

Another example of a ``bad encryption'' system is one wherein an agent in the field has to lug around a 10lb phone book to encrypt her messages.  She's going to have a hard time moving quickly or being inconspicuous. 

We began thinking about the flaws in encryption systems that work as in Figure~\ref{fig:flawed}.

\begin{figure}[!h]\label{fig:flawed}
\centering
\[
\xymatrix{ 
             & & & \text{Odette} \ar[dd]_{intercept} \ar[ddrr]^{intercept} & & \\
\text{Alice} & & &     &          & \text{Bob} \\
{\substack{\text{``Let's see\quad } \\ \text{other people'' \quad}}} \ar[rr]_{encrypt}       &      & \ar[r]_{encrypt\,key} & \ar[rr]_{encrypt\, key's\, key}  &               & {\substack{\text{\quad ``Let's see } \\ \text{\quad other people''} \\ \text{\qquad + plaintext key}}}
}
\]
\caption{Alice sends a message to Bob.  But Bob needs the key to decrypt the message.  To securely send the key to decrypt the message, she encrypts that, too.  Etc.}
\end{figure}

\section{Cracking codes}

To decrypt the course syllabus, Jeff used frequency analysis to try to crack the cipher text.  For future reference, below is a frequency analysis table (Table~\ref{tab:freqanalysis}) for the English language (as presented in \cite{Singh}).  We'll defer further frequency analysis until a bit later in the course.

\begin{table}[!h]\label{tab:freqanalysis}
\centering
\caption{Frequency tables for the English language.}
\begin{footnotesize}
\begin{tabular}{| c | c | c | c || c | c |}
\hline
Letter & Frequency & Letter & Frequency & Group & Range\\
\hline
a & 8.2\%   & n & 6.7\% & e                 & $>$12\% \\
b & 1.5\%   & o& 7.5\%  & t,a,o,i,n,s,h,r   & 6---9\%\\
c & 2.8\%   & p & 1.9\% & d,l               & 4\% \\
d & 4.3\%   & q & 0.1\% & c,u,m,w,f,g,y,p,b & 1.5---3\% \\
e & 12.7\%  & r & 6.0\% & v,k,j,x,q,z       & $<$1\% \\
f & 2.2\%   & s &6.3\%  &&\\
g & 2.0\%   & t &9.1\%  &&\\
h & 6.1\%   & u &2.8\%  &&\\
i & 7.0\%   & v &1.0\%  &&\\
j & 0.2\%   & w &2.4\%  &&\\
k & 0.8\%   & x &0.2\%  &&\\
l & 4.0\%   & y & 2.0\% &&\\
m & 2.4\%   & z & 0.1\% &&\\
\hline
\end{tabular}

In the ``Group'' column, the entries are ordered left to right in decreasing frequency.  E.g., \plain{t} appears with relative frequency 9.1\%, \plain{a} appears with relative frequency 8.2\%, and so on.
\end{footnotesize}
\label{tab:freq-table}
\end{table}

\section{Numbers and Operations in other bases}

Most of us already have an idea of what the expression \[
(110101)_2\]
represents.  Indeed, we interpret it as follows:\[
(110101)_2 = 1\cdot 2^5 + 1\cdot 2^4 + 0\cdot2^3 +1 \cdot 2^2 + 0\cdot 2^1 + 1\cdot 2^0 = 57.
\]

\begin{definition} Let $n \in \ZZ$, $b \in \ZZ_{> 1}$.  Then there exist unique integers $d_0, d_1, \ldots, d_{k-1} \in \{0, 1, 2, \ldots, b-1\}$ such that \[
n = d_{k-1} b^{k-1} + d_{k-2}b^{k-2} + \cdots + d_{2} b^2 + d_1 b + d_0.\]

The integers $d_0, \ldots, d_{k-1}$ are called \defi{digits}, and if $d_{k-1} \not = 0$, $n$ is called a \defi{$k$-digit base-$b$ number}, denoted
\[ n = \left(d_{k-1} d_{k-2} \cdots d_1 d_0\right)_b.\]
When $b = 10$, we drop this notation.  When $b = 2$ is clear from context, we also drop this notation.
\end{definition}

\begin{proposition}\label{prop:!digits} The digits are indeed unique.\end{proposition}

In order to prove this proposition, we need to invoke the following result about integers.

\begin{theorem}[The Division Algorithm] Let $n \in \ZZ$ and $b \in \ZZ_{> 1}$.  Then there are unique integers $m$ and $d$ such that $0 \leq d \leq b-1$ and \[n = m q + d.\]
\end{theorem}

The key observation is that the division algorithm ensures that $d$ is a base-$b$ digit.  Although $m$ may not be a base-$b$ digit, we can apply the division algorithm to $m$ to find the next base-$b$ digit, $d_1$, in the representation of $n$.

\begin{proof}[Proof of \ref{prop:!digits}] Let $n \in \ZZ$, $b \in \ZZ_{> 1}$.  By the division algorithm, there exist $n_0 \in \ZZ$ and $d_{0} \in \ZZ$ where $0 \leq d_0 \leq b-1$ and \begin{equation}\label{eq:!digits1}
n = n_0 b + d_0.
\end{equation}

Similarly, there exists $n_1$ and $d_1$ such that \begin{equation}
n_0 = n_1 b + d_1.
\end{equation}

Repeating this process for each nonzero $n_i$, we eventually find that there is a positive integer $k$ for which \begin{equation}
n_k = 0 \cdot b + d_{k-1}.
\end{equation}

Then substituting these equalities into Equation~\eqref{eq:!digits1}, we obtain \[
n = d_{k-1} b^{k-1} + \cdots + d_2 b^2 + d_1 b + d_0.
\]
By the uniqueness of each $n_i$ in the division algorithm, it follows that this representation of $n$ as a base-$b$ number is unique.
\end{proof}

This is an example of a \defi{constructive proof}.  That means that in order to find the base-$b$ representation of an integer $n$ for given $b$ and $n$, we can follow the steps in the proof.

\begin{example} Working with different bases.
\begin{enumerate}[(1)]
\item Convert $160$ to base-$7$.

We start with long division to find the first remainder, which will be $d_0$.
\[\longdiv{160}{7}\]

At this point, we know $160 = 22\cdot 7 + 6$.

Since $22 \not = 0$, we repeat the process.
\[\longdiv{22}{7}\]

Now we have $22 = 3 \cdot 7 + 1$.  Finally, $3 \div 7$ gives us a quotient of $0$, so we can stop this process.  Back-substitution yields
\[160 = (3\cdot 7 + 1)\cdot 7 + 6 = 3\cdot 7^2 + 1 \cdot 7 + 6 = (316)_7.\]

\item Verify that $(11001001)_2 = 201$.

\item When working in base-$26$, we use the letters $A$ through $Z$ as the digits:

\begin{table}[!h]\label{tab:base26digits}
\centering
\begin{small}
\begin{tabular}{|c|c|c|c|c|c|c|c|c|c|c|c|c|}
\hline
 A & B & C & D & E & F & G & H & I& J & K & L &M  \\
 \hline
0 & 1& 2 & 3 & 4 & 5 & 6 & 7 & 8 & 9 & 10 & 11 & 12 \\
\hline
\hline
N&O&P&Q&R&S&T&U&V&W&X&Y&Z\\
\hline
13 & 14 & 15 & 16 & 17 & 18 & 19 & 20 & 21 & 22 & 23 & 24 & 25\\
\hline
\end{tabular}
\caption{The alphabet as the digits for $base$-26.}
\end{small}
\end{table}

Now we can convert text to integers! \[
(MATH)_{26} = 12\cdot 26^3 + 0 \cdot 26^2 + 19\cdot26 + 7 = 211,413 \]

\item We can also work with decimal/rational numbers in other bases, but we don't necessarily have unique representations in this case.

\[(MA.TH)_{26} = 12^\cdot 26 + 0 + 19 \cdot 26^{-1} + 7 \cdot 26^{-2} = 312 + \frac{510}{676}\]

You can prove the following equality with geometric series.
\[(.ZZZ\overline{Z})_{26} = (B)_{26}\]

\end{enumerate}
\end{example}

\section{Homework}

\begin{enumerate}
\item Read pages 1 - 44 in \cite{Singh}.  Take note of the definitions of italicized words.  We'll discuss them on the first day of class.

\item (Foiling Cryptanalysis) Study pp.\ 32-44 of Simon Singh's book.  Draft a short summary (1 paragraph) of Babington's plot without using ``e'' (-1 point for any occasion that this glyph shows up in your writing). As usual, grammar counts! You may consult a dictionary or similar book to assist your writing.  Follow our format for class writing.  Good luck!
\item Perform the following calculations.
\begin{enumerate}
\item Multiply $(212)_3$ by $(122)_3$.
\item Divide $(40122)_7$ by $(126)_7$.
\item Multiply the binary numbers $101101$ and $11001$, and divide $10011001$ by $1011$.
\item 3 points. In base 26 with digits $A - Z$ representing $0 - 25$, multiply $YES$ by $NO$, and divide $JQVXHJ$ by $WE$.
\end{enumerate}
\end{enumerate}

