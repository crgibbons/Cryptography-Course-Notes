
%CHAPTER 13 BEGINS HERE (Beth C)
\begin{chapquote}{Professor Gibbons}
``Whenever I say "Elliptic Diffie-Hellman", I picture Whitfield Diffie on an elliptical.''
\end{chapquote}
\begin{chapquote}{Professor Gibbons}
''I'm going to kill my cat.''
\end{chapquote}

\section{Digital Signatures}
Digital signatures can be used to authenticate the identity of a message's author.

Overview of Digital Signatures:
\begin{itemize}
\item $k^{priv}$: private signing key
\item $k^{pub}$: public verification key
\item "sign": an algorithm that uses $k^{priv}$ to "sign" digital message M. 
\item "verify": algorithm using $k^{pub}$ to verify that $M^{sig}$ matches M.
\end{itemize}

Goals of Digital Signatures:
\begin{itemize}
\item 1. Create a system where $k^{priv}$ remains private despite $k^{pub}$ being public
\item 2. Keep $k^{priv}$ private despite publishing many signature pairs.
\end{itemize}

RSA Digital Signatures:
Suppose Alice is creating a digital signature keypair. Pick 2 large primes, $p$ and $q$, and calculate $n=pq$ and $\phi(n)=(p-1)(q-1)$. Randomly choose $e$ where $gcd(\phi(n),e)=1$. Next solve the diophantine equation $ed+\phi(n)k=1$ to find private key $d$.


\begin{itemize}
\item Public information: (n,e) is $k^{pub}$
\item Private information: d is $k^{priv}$
\end{itemize}

Alice signs her message M: $M^{d}\mod{n}$ = $M^{sig}$ and sends the pair $(M, M^{sig}$) to Bob.

To verify that the message M was authored by Alice, Bob calculates $(M^{\text{sig}})^{e}\mod{n}$. If $M^{ed}\mod{n} = M$, then Bob knows Alice wrotes the message. (This works as long as $1 \leq M < n$ and $M \ne p$, $M \ne q$.)

\begin{example}
Alice and Bob use Diffie-Hellman to create a shared secret key. Alice will also include a digital signature.
\begin{itemize}
\item Diffie-Hellman primes: $p=13$, $q=2$
\item Alice's secret integer: $a=5$
\item Digital message to Bob: $ M = q^{a}\mod{p} = 2^{5}\mod{13} = 6$
\end{itemize}

Create $M^{sig}$
\begin{itemize}
\item Public key: $n = 14417$, $e = 17$
\item Private key: $d = 6257$
\item $M^{sig} = 6^{6257} \mod{14417 = 10753}$

\end{itemize}

Bob receives (6, 10753) and calculates:
$10753^{17}\mod{14417} = 6$.
Since $M = (M^{sig})^{e} \mod{n}$, Bob knows the message is from Alice. 
\end{example}

Goals: 
\begin{itemize}
\item $k^{priv}$ remains private even though $k^{pub}$ is public.\\
Note: RSA Digital Signatures are as secure (99.8\%) as RSA encryption.
\item $k{priv}$ remains private even with A LOT of public message, signature pairs.\\
Note: We don't know yet.
\end{itemize}




On Wednesday 4/29 we had an in-class quiz during which we attempted to construct ways to attack the security of RSA Digital Signature Encryption with the acquistion of many message-signature pairs.\\
\section{Security of RSA Signatures}

\underline{Claim}: We can't find $d$ in a timely fasion. 

Brute force technique: try to find all possible $d$.
In their original paper (\textit{Rivest, Shamir, Adleman, 1978}) recommend $p$ and $q$ are at least 200 digits each so that $n$ is about 400 digits long. The largest 400 digit number is equal to $10^{400}-1$.\\

The authors show that any "conceivable" way to find $d$ is at least as computationally as hard as factoring $n$.\\ 
Back to RSA Cryptography (as opposed to digital signatures):\\
Finding $d$:
\begin{itemize}
\item Factor $n$. But this can take a very long time.
For example:
\begin{table}[ht]
\caption{Factoring Large Numbers}
\centering
\begin{tabular}{c c c}
\hline \hline
Digits & Number of Operations & Time (1978) \\ [.5ex]
\hline
50 & 1.4 x $10^{10}$ & 4 hours \\
75 & 9 x $10^{12}$ & 104 days \\
100 &  & 74 years \\
500 &  & 4.2 x $10^{25}$ years \\ [1ex]
\hline
\end{tabular}
\end{table}

\item Find $d$ without factoring $n$ or computing $\phi(n)$.\\This can be done by finding the multiplicative inverse of $e$ modulo $n$.
\end{itemize}
\underline{Claim}: This is as hard as factoring $n$ because it leads to an algorithm for factoring $n$ (which are all slow).
If we find $d$, we know $$(ed)-1 \equiv \phi(n)k, \text{ for some k}.$$

A result of Miller: knowing any multiple of $\phi(n)$ leads to a factorization algorithm for $n$.\\

Given a lot of message-signature pairs, is RSA digital secure? 

\underline{Caveat}: We need $gcd(M_i, n) = 1$ (Euler's Theorem). Each $M_i$ gives us non-factors of $n$. Can we accidentally eliminate all non-factors of $n$ and be left with a small list of factors?

\underline{How to avoid this}: Consider $n = pq$ where $p$ and $q$ are primes of around 200 digits. If we force $M_i$ to have fewer than 200 digits, we can't eliminate any 200-digit primes. Thus, factoring $n$ is still hard. To rule out all odd numbers with 200 digits, we would need around $45 \times 10^{198}$ message pairs.

\begin{proposition}
Finding $d$ in the RSA digital signature scheme (where $M_i$ has fewer than 200 digits) leads to a factorization algorithm for $n$.
\end{proposition}
\begin{proof}
We proceed by Induction on the number of message-signature pairs. 
Let $k = 0$. Then, $M_i^{de}$ \% $n = M_i$ and I know $M$, $e$, $M^d$, and $n$. Finding $d$ is equivalent to finding $\varphi(n)k$ for some $k \in \ZZ$ in, $$ ed-1 = \varphi(n)k.$$ By Miller's result, this allows us to factor $n$.
Let $k = 2$. Then we have $(M_1, M_1^d)$ and $(M_2, M_2^d)$. What I know can be summarized in the equation $$((M_1 + M_2)^2)^{ed}  \text{ \% }  n = (M_1 + M_2)^2  \text{ \% }  n.$$ This is a single result that can be put into Miller's result which allows us to factor $n$. 
\end{proof}
