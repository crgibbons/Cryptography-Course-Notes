\documentclass[letter,11pt]{book}
\usepackage[T1]{fontenc}
\usepackage[utf8]{inputenc}
\usepackage{lmodern}
\usepackage[acronym,toc,nonumberlist]{glossaries}
\usepackage{hyperref}
\usepackage{graphicx}
\usepackage[english]{babel}
\usepackage{rotchiffre}
\usepackage{xypic}
\usepackage{amsmath,amsthm,amssymb}
\usepackage{enumerate}
\usepackage{tikz}

%%%%%%%%%%%
% Math Macros
%%%%%%%%%%%%
\newcommand{\ZZ}{\mathbb{Z}}

%%%%%%%%%%%%%%%%%%%%%%%%%%%%%%%%%%%%%%%%%%%%%%%%%%%%%%%%%%%
% Distinguished styles for text
%%%%%%%%%%%%%%%%%%%%%%%%%%%%%%%%%%%%%%%%%%%%%%%%%%%%%%%%%%%
\newcommand{\plain}[1]{{\tt\lowercase{#1}{}}}
\newcommand{\cipher}[1]{{\tt\uppercase{#1}{}}}
\newcommand{\defi}[1]{\textbf{\textit{#1}}{}}
\renewcommand*{\glstextformat}{\defi}

%%%%%%%%%%%
% Standard Math Environments
%%%%%%%%%%%%%%%%%%
\theoremstyle{definition}
\newtheorem{definition}{Definition}[chapter]
\newtheorem{example}{Example}[chapter]
\newtheorem{problem}{Problem}[chapter]

\newcommand{\solution}{\textbf{Solution.\quad}}

\theoremstyle{theorem}
\newtheorem{theorem}{Theorem}[chapter]
\newtheorem{proposition}{Proposition}[chapter]
\newtheorem{corollary}{Corollary}[chapter]

%%%%%%%%%%%%%%%%%%%%%%%%%%%%%%%%%%%%%%%%%%%%%%%%%%%%%%%%%%%%%%%%%%%%%%%%%%%%%%%%
% 'dedication' environment: To add a dedication paragraph at the start of book %
% Source: http://www.tug.org/pipermail/texhax/2010-June/015184.html            %
%%%%%%%%%%%%%%%%%%%%%%%%%%%%%%%%%%%%%%%%%%%%%%%%%%%%%%%%%%%%%%%%%%%%%%%%%%%%%%%%
\newenvironment{dedication}
{
   \thispagestyle{empty}
   \vspace*{\stretch{1}}
   \hfill\begin{minipage}[t]{0.66\textwidth}
   \raggedright
}
{
   \end{minipage}
   \vspace*{\stretch{3}}
   \clearpage
}

%%%%%%%%%%%%%

%%%%%%%%%%%%%

%%%%%%%%%%%%%%%%%%%%%%%%%%%%%%%%%%%%%%%%%%%%%%%%
% Chapter quote at the start of chapter        %
% Source: http://tex.stackexchange.com/a/53380 %
%%%%%%%%%%%%%%%%%%%%%%%%%%%%%%%%%%%%%%%%%%%%%%%%
\makeatletter
\renewcommand{\@chapapp}{}% Not necessary...
\newenvironment{chapquote}[2][2em]
  {\setlength{\@tempdima}{#1}%
   \def\chapquote@author{#2}%
   \parshape 1 \@tempdima \dimexpr\textwidth-2\@tempdima\relax%
   \itshape}
  {\par\normalfont\hfill--\ \chapquote@author\hspace*{\@tempdima}\par\bigskip}
\makeatother

%%%%%%%%%%
% Load up the Glossary entries
%%%%%%%%%%
\makeglossaries
\newglossaryentry{steganography}
{
    name=steganography,
    description={the practice of writing secret communication achieved by hiding the evidence of a message}
}

\newglossaryentry{cryptography}
{
    name=cryptography,
    description={the practice of hiding the meaning of a message}
}

\newglossaryentry{encryption}
{
    name={encryption},
    description={the process of hiding a message by scrambling it according to a protocol}
}

\newglossaryentry{nulls}
{
    name={nulls},
    description={symbols or letters that are not substitutes for actual letters that were not substitutes for actual letters but blanks that represent nothing}
}

\newglossaryentry{microdot}{
    name={microdot},
    description={an encrypted message shrunk down to a small dot and hidden within another page}
}
    
\newglossaryentry{transposition}
{
    name={transposition},
    description={an encryption protocol that rearranges the letters of the message}
}

\newglossaryentry{substitution}
{
    name={substitution},
    description={an encryption protocol that scrambles a message by substituting symbols (often letters) for parts of the plain text}
}

\newglossaryentry{scytale}
{
    name={scytale},
    description={an example of a transposition protocol used by Greek soldiers}
}

\newglossaryentry{plain alphabet}
{
    name={plain alphabet},
    description={alphabet used to write the original (unscrambled) message.  It is also called the ``plaintext alphabet.'' }
}

\newglossaryentry{cipher alphabet}
{
    name={cipher alphabet},
    description={a list of letters used to substitute for the letters in the plain alphabet}
}

\newglossaryentry{Caesar shift}
{
    name={Caesar shift cipher},
    description={an encryption protocol that creates a cipher alphabet by shifting the plaintext alphabet by a fixed number of places}
}

\newglossaryentry{algorithm}
{
    name={algorithm},
    description={ }
}

\newglossaryentry{key}
{
    name={key},
    description={enough information to decrypt a message if one knows the encryption protocol}
}

\newglossaryentry{keyword}
{
    name={keyword},
    description={a word used to create a cipher alphabet}
}

\newglossaryentry{key phrase}
{
    name={key phrase},
    description={like a keyword, but a phrase}
}

\newglossaryentry{monoalphabetic substitution cipher}
{
    name={monoalphabetic substitution cipher},
    description={ }
}

\newglossaryentry{cryptanalysis}
{
    name={cryptanalysis},
    description={ }
}

\newglossaryentry{frequency analysis}
{
    name={frequency analysis},
    description={the practice of analyzing the frequency of letters appearing in encrypted text in order to reveal the contents of the message}
}

\newglossaryentry{code}
{
    name={code},
    description={an encryption protocol that makes substitutions at the level of words}
}

\newglossaryentry{cipher}
{
    name={cipher},
    description={an encryption protocol that makes substitutions at the level of letters}
}

\newglossaryentry{encipher}
{
    name={encipher},
    description={to scramble a message using ciphers}
}

\newglossaryentry{decipher}
{
    name={decipher},
    description={to unscramble a ciphered message using a key}
}

\newglossaryentry{encode}
{
    name={encode},
    description={to scramble a message using codes}
}

\newglossaryentry{decode}
{
    name={decode},
    description={to unscramble a coded message using a key}
}

\newglossaryentry{encrypt}
{
    name={encrypt},
    description={to scramble a message according to an encryption protocol}
}

\newglossaryentry{decrypt}
{
    name={encrypt},
    description={to unscramble a decrypted message using a key}
}

\newglossaryentry{nomenclator}
{
    name={nomenclator},
    description={a system of encryption that relies on a cipher alphabet, which is used to encrypt the majority of a message, and a limited list of codewords}
}

\newglossaryentry{Kerchoff's Principle}{
    name={Kerchoff's Principle},
    description={the philosophy that the security of an encryption system must not depend on keeping secret the encryption algorithm but keeping secret the key}
}




%%% Blank.  Copy/paste and uncomment to use
%\newglossaryentry{}
%{
%    name={ },
%    description={ }
%}



%%%%%%%%%
% Long Division
%%%%%%%%%

%  longdiv.tex  v.1  (1994)  Donald Arseneau  
%
%  Work out and print integer long division problems.  Use:
%       \longdiv{numerator}{denominator}
%  The numerator and denominator (divisor and dividend) must be integers, and
%  the quotient is an integer too.  \longdiv leaves a remainder.
%  Use this in any type of TeX.

\newcount\gpten % (global) power-of-ten -- tells which digit we are doing
\countdef\rtot2 % running total -- remainder so far
\countdef\LDscratch4 % scratch

\def\longdiv#1#2{%
 \vtop{\normalbaselines \offinterlineskip
   \setbox\strutbox\hbox{\vrule height 2.1ex depth .5ex width0ex}%
   \def\showdig{$\underline{\the\LDscratch\strut}$\cr\the\rtot\strut\cr
       \noalign{\kern-.2ex}}%
   \global\rtot=#1\relax
   \count0=\rtot\divide\count0by#2\edef\quotient{\the\count0}%\show\quotient
   % make list macro out of digits in quotient:
   \def\temp##1{\ifx##1\temp\else \noexpand\dodig ##1\expandafter\temp\fi}%
   \edef\routine{\expandafter\temp\quotient\temp}%
   % process list to give power-of-ten:
   \def\dodig##1{\global\multiply\gpten by10 }\global\gpten=1 \routine
   % to display effect of one digit in quotient (zero ignored):
   \def\dodig##1{\global\divide\gpten by10
      \LDscratch =\gpten
      \multiply\LDscratch  by##1%
      \multiply\LDscratch  by#2%
      \global\advance\rtot-\LDscratch \relax
      \ifnum\LDscratch>0 \showdig \fi % must hide \cr in a macro to skip it
   }%
   \tabskip=0pt
   \halign{\hfil##\cr % \halign for entire division problem
     $\quotient$\strut\cr
     #2$\,\overline{\vphantom{\big)}%
     \hbox{\smash{\raise3.5\fontdimen8\textfont3\hbox{$\big)$}}}%
     \mkern2mu \the\rtot}$\cr\noalign{\kern-.2ex}
     \routine \cr % do each digit in quotient
}}}


%%%%%%%%%%%%%%%%%%%%%%%%%%%%%%%%%%%%%%%%%%%%%%%%%%%
% First page of book which contains 'stuff' like: %
%  - Book title, subtitle                         %
%  - Book author name                             %
%%%%%%%%%%%%%%%%%%%%%%%%%%%%%%%%%%%%%%%%%%%%%%%%%%%

% Book's title and subtitle
\title{\Huge \textbf{Cryptography}  \\ \huge Course Notes, Spring 2015 % Author
\author{\textsc{Courtney Gibbons (\& class)}}}

